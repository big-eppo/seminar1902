\documentclass[twoside,a4paper,12pt]{article}

\usepackage{EthischeFragenBeimAutomatisiertenFahren}

\begin{document}

\thispagestyle{empty}
\begin{center}
	\includegraphics[scale=1]{resources/fernunisignet-sw}
\end{center}
\vskip 4cm
\begin{center}
	{\Large Seminararbeit 1902 WS 2018/19\par}
	\vskip 2.5cm
	{\textbf{\Large Ethische Fragen} \par}    
	\vskip 0.2cm	
	{\textbf{\Large beim automatisierten Fahren} \par}    
	\vskip 1cm
	{\Large Nikolai Kabiolskij\\Christian Oevermann \par}
	\vskip 4.5 cm
	\Large Januar 2019
\end{center}
\vspace*{\stretch{1}}
\FUiH\\
Fakultät für Mathematik und Informatik\\
Lehrgebiet Kooperative Systeme\\
Betreut von apl. Prof. Dr. Christian Icking und Dr. Lihong Ma\\
  
\newpage
\shipout\hbox{}
\newpage

\frontmatter

\section{Einleitung} \label{Einleitung}
Seit je und immer wollten die Menschen sich fortbewegen können und haben Ihre Mobilität zunehmend ausgebaut. Mit dem rasanten Anstieg von Technologien und der zunehmenden Digitalisierung in der Gesellschaft entstehen neue Möglichkeiten, welche man für die Mobilität der Zukunft einsetzen kann und soll. Das Thema dieser Arbeit ist das autonome Fahren, im Konkreten werden die ethischen Fragen, welche dabei entstehen erläutert. \\ Auch wenn das autonome Fahren als sicher zu erscheinen vermag, da hierfür nur sichere Systeme eingesetzt werden sollen, ist die Möglichkeit eines Unfalls dennoch nicht ausgeschlossen. Es können ebenso verschiedene Dilemma Situationen auftreten bei denen keine der möglichen Entscheidungen hundertprozentig richtig ist. Ebenso ist die Richtigkeit der Entscheidung von Mensch zu Mensch unterschiedlich und verleiht uns allen eine besondere Persönlichkeit. \\ Um all die ethischen Fragen rund um autonomes Fahren zu beantworten oder zumindest anzugehen wurde von dem damaligen Bundesverkehrsminister Herr Alexander Dobrindt eine Ethik-Kommission zum automatisierten Fahren eingesetzt. Geleitet wurde die Kommission von dem ehemaligen Bundesverfassungsrichter Prof. Dr. Dr. Udo Di Fabio. Die Ethik-Kommission des BMVI setzte sich aus 14 Wissenschaftlern und Experten aus den Fachrichtungen Ethik, Recht und Technik zusammen. \cite{site1} Ziel der Kommission war, die Leitlinien für die Entwicklung von autonom fahrenden Systemen zu erarbeiten. \\
Das Ergebnis der fünf Sitzungen der Kommission sind, die 20 Ethische Regeln für den automatisierten
und vernetzten Fahrzeugverkehr, welche in dieser Arbeit näher betrachtet werden. 

\newpage

\tableofcontents

\newpage

\mainmatter

\cleardoublepage
\section{Vorstellung und Diskussion der ethischen Regeln} \label{VorstellungUndDiskussionDerEthischenRegeln}

Im nachfolgenden Hauptteil dieser Ausarbeitung werden jeweils die von der Kommission formulierten ethischen Regeln zitiert und im Anschluss einer kritischen 
Würdigung hinsichtlich ihrer technischen Implementierbarkeit unterzogen. Diese fällt i.\,d.\,R. wegen der gebotenen Kürze knapp aus und ist natürlich
von der subjektiven Meinung bzw. dem Assoziationsraum der Autoren geprägt. 

\subsection{Primärziel Sicherheit} \label{PrimaerzielSicherheit}

\begin{quote}
\glqq
Teil- und vollautomatisierte Verkehrssysteme dienen zuerst der Verbesserung der Sicherheit aller Beteiligten im Straßenverkehr. 
Daneben geht es um die Steigerung von Mobilitätschancen und die Ermöglichung weiterer Vorteile. Die technische Entwicklung 
gehorcht dem Prinzip der Privatautonomie im Sinne eigenverantwortlicher Handlungsfreiheit.\grqq\mbox{~\cite[S. 10]{ek}}
\end{quote}
Die autonom fahrende und vernetzte Fahrzeuge sind nur dann ethisch vertretbar, wenn sie die Unfallwahrscheinlichkeit gegenüber menschlichen Fahrern verringern und zu dem allgemeinem Verkehrsfluss positiv beitragen. Weiterhin müssen die autonomen Fahrzeuge so konzipiert werden, dass sie auch auf andere Verkehrsteilnehmer wie Fußgänger, Radfahrer aber auch nicht autonom fahrende Autos  einstellen. Die Systeme müssen bereit sein auf diverse nicht vorhersehbare Situationen reagieren zu können und das Risiko eines Unfalls minimieren zu können. Hier geht die Sicherheit klar vor dem Komfort, nicht mehr selbst fahren zu müssen und sich den anderen Dingen zu widmen. Dennoch soll dieser Punkt nicht ganz ausgelassen werden, da die autonome Mobilität außer Sicherheit viele weitere Vorteile in sich hat. Die vernetzten und autonome Fahrzeuge könnten den Verkehr flüssiger machen und insgesamt das Fahren in mehr dicht bewohnten Städten angenehmer gestalten. Es ist durchaus denkbar, dass die Systeme in Autos den Platz optimal ausnutzen und weniger Staus verursachen. Trotz der vielen Chancen und Vorteilen des autonomen Fahrens müssen die Risikoaspekten besonders im gemischten Verkehr betrachtet  und die Haftungsfragen neu beantwortet werden. \\  Gleichzeitig in der selben Leitlinie möchte uns die Kommission zu verstehen geben, dass die potentielle Nutzer eigenverantwortlich selbst entscheiden dürfen, ob und wie sie das autonome Fahren verwenden möchten. Solche Entscheidung dürfen keineswegs eingeschränkt oder gar aufgezwungen werden.
\newpage
\subsection{Vorrang des Schutzes von Menschen} \label{VorrangDesSchutzesVonMenschen}

\begin{quote}
\glqq
Der Schutz von Menschen hat Vorrang vor allen anderen Nützlichkeitserwägungen. Ziel
ist die Verringerung von Schäden bis hin zur vollständigen Vermeidung. Die Zulassung
von automatisierten Systemen ist nur vertretbar, wenn sie im Vergleich zu menschlichen
Fahrleistungen zumindest eine Verminderung von Schäden im Sinne einer positiven Risikobilanz verspricht.\grqq\mbox{~\cite[S. 10]{ek}}
\end{quote}
Die Entwicklung von Systemen zum autonomen Fahren soll stets das Ziel, das menschliche Leben zu schützen, als primäres Ziel im Auge behalten. Alle weitere bereits besprochene Ziele, wie Komfort oder verbesserter Verkehrsfluss o.ä. sind allesamt sekundär und dürfen keineswegs vorangestellt werden. Diese Ansätze müssen bereits im Kern der Implementierung angesiedelt sein und stets weiterverfolgt werden. Es ist nur unter diesen Umständen ethisch vertretbar solche Systeme überhaupt zuzulassen und massenhaft zu betreiben. \\ Gleichwohl ist der Schutz ein sehr umfassendes Thema und wurde natürlich von der Kommission viel breiter diskutiert. Weitere Punkte  befinden sich in einem  \hyperlink{target1}{weiteren} Paragraph dieser Arbeit.


\subsection{Gewährleistungsverantwortung der öffentlichen Hand} \label{GewaehrleistungsverantwortungDerOeffentlichenHand}

\begin{quote}
\glqq
Die Gewährleistungsverantwortung für die Einführung und Zulassung automatisierter
und vernetzter Systeme im öffentlichen Verkehrsraum obliegt der öffentlichen Hand.
Fahrsysteme bedürfen deshalb der behördlichen Zulassung und Kontrolle. Die Vermeidung von Unfällen ist Leitbild, wobei technisch 
unvermeidbare Restrisiken einer Einführung des automatisierten Fahrens bei Vorliegen einer grundsätzlich positiven Risikobilanz
nicht entgegenstehen.\grqq\mbox{~\cite[S. 10]{ek}}
\end{quote}
So wie heute alle Fahrzeuge in Deutschland einer Haupt- und Abgasuntersuchung unterliegen, müssen ebensolche Maßnahmen für autonom fahrende Autos geschaffen werden. Dabei gilt zu beachten, dass in der heutigen globalisierten Welt, zunächst bestimmte Normen erschaffen werden  müssen, um die Definition von einem sicheren autonomen Fahrzeug allgemein gleich zu halten. Für die Ausarbeitung solcher Normen wären die Regierungen einzelner Staaten der EU und der restlichen Welt zuständig. Dabei muss man bedenken, dass die Vereinheitlichung aller Normen ebenso vorgenommen werden muss. So wie einst das ECE-Prüfzeichen erschaffen wurde, wäre eine einheitliche europäische Kennzeichnung von geprüften Systemen denkbar.

\subsection{Entscheidungsfreiheit des Einzelnen} \label{EntscheidungsfreiheitDesEinzelnen}

\begin{quote}
\glqq
Die eigenverantwortliche Entscheidung des Menschen ist Ausdruck einer Gesellschaft, in
der der einzelne Mensch mit seinem Entfaltungsanspruch und seiner Schutzbedürftigkeit
im Zentrum steht. Jede staatliche und politische Ordnungsentscheidung dient deshalb
der freien Entfaltung und dem Schutz des Menschen. In einer freien Gesellschaft erfolgt
die gesetzliche Gestaltung von Technik so, dass ein Maximum persönlicher Entscheidungsfreiheit in einer allgemeinen 
Entfaltungsordnung mit der Freiheit anderer und ihrer
Sicherheit zum Ausgleich gelangt.\grqq\mbox{~\cite[S. 10]{ek}}
\end{quote}

\subsection{Unfallvermeidung} \label{Unfallvermeidung}

\begin{quote}
\glqq
Die automatisierte und vernetzte Technik sollte Unfälle so gut wie praktisch möglich vermeiden. Die Technik muss nach 
ihrem jeweiligen Stand so ausgelegt sein, dass kritische
Situationen gar nicht erst entstehen, dazu gehören auch Dilemma-Situationen, also eine
Lage, in der ein automatisiertes Fahrzeug vor der \glqq Entscheidung\grqq\ steht, eines von zwei
nicht abwägungsfähigen Übeln notwendig verwirklichen zu müssen. Dabei sollte das gesamte Spektrum technischer 
Möglichkeiten --- etwa von der Einschränkung des Anwendungsbereichs auf kontrollierbare Verkehrsumgebungen, 
Fahrzeugsensorik und Bremsleistungen, Signale für gefährdete Personen bis hin zu einer Gefahrenprävention mittels
einer \glqq intelligenten\grqq\ Straßen-Infrastruktur --- genutzt und kontinuierlich weiterentwickelt
werden. Die erhebliche Steigerung der Verkehrssicherheit ist Entwicklungs- und Regulierungsziel, und zwar bereits in der 
Auslegung und Programmierung der Fahrzeuge zu defensivem und vorausschauendem, schwächere Verkehrsteilnehmer (\glqq Vulnerable Road
Users\grqq) schonendem Fahren.\grqq\mbox{~\cite[S. 10]{ek}}
\end{quote}

\subsection{Abwägungen zur Einsatzpflicht} \label{AbwaegungenZurEinsatzpflicht}

\begin{quote}
\glqq
Die Einführung höherer automatisierter Fahrsysteme insbesondere mit der Möglichkeit
automatisierter Kollisionsvermeidung kann gesellschaftlich und ethisch geboten sein,
wenn damit vorhandene Potentiale der Schadensminderung genutzt werden können.
Umgekehrt ist eine gesetzlich auferlegte Pflicht zur Nutzung vollautomatisierter Verkehrssysteme oder die Herbeiführung 
einer praktischen Unentrinnbarkeit ethisch bedenklich, wenn damit die Unterwerfung unter technische Imperative verbunden 
ist (Verbot der Degradierung des Subjekts zum bloßen Netzwerkelement).\grqq\mbox{~\cite[S. 11]{ek}}
\end{quote}

\subsection{Priorität des Schutzes menschlichen Lebens} \label{PrioritäDesSchutzesMenschlichenLebens}


\begin{quote}
\glqq
\hypertarget{target1}
In Gefahrensituationen, die sich bei aller technischen Vorsorge als unvermeidbar erweisen, besitzt der Schutz menschlichen 
Lebens in einer Rechtsgüterabwägung höchste Priorität. Die Programmierung ist deshalb im Rahmen des technisch Machbaren 
so anzulegen, im Konflikt Tier- oder Sachschäden in Kauf zu nehmen, wenn dadurch Personenschäden vermeidbar sind.\grqq\mbox{~\cite[S. 11]{ek}}
\end{quote}
In Gefahrensituationen genießt der Schutz des menschlichen Lebens eine höchste Priorität. So hat es zu bedeuten, dass im Falle eines Unfalls Sachschäden, wenn möglich, in Kauf genommen werden müssen um Menschenleben zu retten. Diese Aussage scheint auf den ersten Blick ethisch vollständig korrekt zu sein, jedoch können dabei Nachfolgekomplikationen auftreten. So hat die ethische Kommission ein Beispiel aufgezeichnet in dem folgende Situation beschrieben wird. \begin{quote}Die Folge eines Sachschadens das Auslaufen eines Tanklasters oder auch der Zusammenbruch des Stromnetzes einer Metropolregion sein könnte. \mbox{~\cite[S.17]{ek}}\end{quote}
Das Beispiel zeigt, dass manche Sachschäden enorme Auswirkungen mit sich bringen können, welche unter Umständen zu weiteren Personenschäden führen könnten. Daher ist der Leitsatz "Sachschäden gehen vor Personenschäden" nicht immer eindeutig zu beurteilen. In diesem Punkt wäre eine mögliche Folgenabschätzung ein Teil des Systems sein können um die Personenschäden zu minimieren. Es soll auch erwähnt werden, dass zu Einem, es nicht möglich sein wird, alle Szenarien der Maschine beizubringen und zum anderen die Menschen selbst sind niemals in der Lage alle möglichen Ausgänge eines Unfalls in kürzesten Zeit zu zu analysieren und entscheiden sich meist intuitiv. 

\subsection{Nicht-Normierbarkeit dilemmatischer Entscheidungen} \label{NichtNormierbarkeitDilemmatischerEntscheidungen}

\begin{quote}
\glqq
Echte dilemmatische Entscheidungen, wie die Entscheidung Leben gegen Leben sind von
der konkreten tatsächlichen Situation unter Einschluss \glqq unberechenbarer\grqq\ Verhaltensweisen Betroffener abhängig. 
Sie sind deshalb nicht eindeutig normierbar und auch nicht
ethisch zweifelsfrei programmierbar. Technische Systeme müssen auf Unfallvermeidung
ausgelegt werden, sind aber auf eine komplexe oder intuitive Unfallfolgenabschätzung
nicht so normierbar, dass sie die Entscheidung eines sittlich urteilsfähigen, verantwortlichen Fahrzeugführers ersetzen 
oder vorwegnehmen könnten. Ein menschlicher Fahrer
würde sich zwar rechtswidrig verhalten, wenn er im Notstand einen Menschen tötet, um
einen oder mehrere andere Menschen zu retten, aber er würde nicht notwendig schuldhaft handeln. Derartige in der Rückschau 
angestellte und besondere Umstände würdigende Urteile des Rechts lassen sich nicht ohne weiteres in abstrakt-generelle 
Ex-Ante Beurteilungen und damit auch nicht in entsprechende Programmierungen umwandeln.
Es wäre gerade deshalb wünschenswert, durch eine unabhängige öffentliche Einrichtung
(etwa einer Bundesstelle für Unfalluntersuchung automatisierter Verkehrssysteme oder
eines Bundesamtes für Sicherheit im automatisierten und vernetzten Verkehr) Erfahrungen systematisch zu verarbeiten.\grqq\mbox{~\cite[S. 11]{ek}}
\end{quote}

\subsection{Verbot der Qualifizierung möglicher Opfer nach persönlichen Merkmalen} \label{VerbotDerQualifizierungMoeglicherOpferNachPersönlichenMerkmalen}

\begin{quote}
\glqq
Bei unausweichlichen Unfallsituationen ist jede Qualifizierung nach persönlichen Merkmalen (Alter, Geschlecht, 
körperliche oder geistige Konstitution) strikt untersagt. Eine
Aufrechnung von Opfern ist untersagt. Eine allgemeine Programmierung auf eine Minderung der Zahl von Personenschäden 
kann vertretbar sein. Die an der Erzeugung von
Mobilitätsrisiken Beteiligten dürfen Unbeteiligte nicht opfern.\grqq\mbox{~\cite[S. 11]{ek}}
\end{quote}

\subsection{Verschiebung der Verantwortung auf die Hersteller} \label{VerschiebungDerVerantwortungAufDieHersteller}

\begin{quote}
\glqq
Die dem Menschen vorbehaltene Verantwortung verschiebt sich bei automatisierten und
vernetzten Fahrsystemen vom Autofahrer auf die Hersteller und Betreiber der technischen Systeme und die infrastrukturellen, 
politischen und rechtlichen Entscheidungsinstanzen. Gesetzliche Haftungsregelungen und ihre Konkretisierung in der gerichtlichen
Entscheidungspraxis müssen diesem Übergang hinreichend Rechnung tragen.\grqq\mbox{~\cite[S. 11]{ek}}
\end{quote}

\subsection{Produkthaftung} \label{Produkthaftung}

\begin{quote}
\glqq
Für die Haftung für Schäden durch aktivierte automatisierte Fahrsysteme gelten die gleichen Grundsätze wie in der übrigen 
Produkthaftung. Daraus folgt, dass Hersteller oder
Betreiber verpflichtet sind, ihre Systeme fortlaufend zu optimieren und auch bereits ausgelieferte Systeme zu beobachten und zu 
verbessern, wo dies technisch möglich und zumutbar ist.\grqq\mbox{~\cite[S. 12]{ek}}
\end{quote}

Beim Thema Produkthaftung muss zwischen den Fahrzeugherstellern und den Betreibern der für automatisiertes Fahren notwendigen
Kommunikationsinfrastruktur differenziert werden. Als dritte Gruppe potentiell Haftender kommen private Anbieter von Datenbank-Systemen z.\,B. für 
Topographie-, Wetter- oder Unfallszenario-Daten in Betracht. Grundsätzlich dürfen Fahrzeugfunktionen, die auf einem gewissen Firmware-Releasestand
eines Fahrzeugs aufsetzen, bzw. auf die Verfügbarkeit einer Kommunikationsinfrastruktur oder die Erreichbarkeit bestimmter externen Dienste
angewiesen sind, nur dann vom Fahrzeug angeboten bzw. genutzt werden, wenn alle \glqq Betriebssystem\grqq -Voraussetzungen dafür erfüllt sind 
und die notwendigen Dienste in ausreichender Qualität zur Verfügung stehen. Dies könnte zur Beweissicherung in einer Unfallsituation
z.\, B. durch eine manipulationssichere Logbuch-Funktion eines Hardware-Gerätes an Bord eines jeden Fahrzeugs ähnlich den Flugschreibern in der Luftfahrt
geschehen. Um den deutschen und europäischen Anforderungen der Datensparsamkeit und -vermeidung gerecht zu werden, könnte diese 
Logbuchfunktion rollierend ausgeführt sein und nur einen überschaubaren Zeitraum abdecken.


\subsection{Aufklärung der Öffentlichkeit} \label{AufklaerungDerOeffentlichkeit}

\begin{quote}
\glqq
Die Öffentlichkeit hat einen Anspruch auf eine hinreichend differenzierte Aufklärung
über neue Technologien und ihren Einsatz. Zur konkreten Umsetzung der hier entwickelten Grundsätze sollten in möglichst 
transparenter Form Leitlinien für den Einsatz und die
Programmierung von automatisierten Fahrzeug+en abgeleitet und in der Öffentlichkeit
kommuniziert und von einer fachlich geeigneten, unabhängigen Stelle geprüft werden.\grqq\mbox{~\cite[S. 12]{ek}}
\end{quote}

\subsection{Ausschluss der totalen Überwachung} \label{AusschlussDerTotalenUeberwachung}

\begin{quote}
\glqq
Ob in Zukunft eine dem Bahn- und Luftverkehr entsprechende vollständige Vernetzung
und zentrale Steuerung sämtlicher Kraftfahrzeuge im Kontext einer digitalen Verkehrsinfrastruktur möglich und sinnvoll sein wird, 
lässt sich heute nicht abschätzen. Eine vollständige Vernetzung und zentrale Steuerung sämtlicher Fahrzeuge im Kontext einer 
digitalen Verkehrsinfrastruktur ist ethisch bedenklich, wenn und soweit sie Risiken einer totalen Überwachung der Verkehrsteilnehmer 
und der Manipulation der Fahrzeugsteuerung nicht sicher auszuschließen vermag.\grqq\mbox{~\cite[S. 12]{ek}}
\end{quote}

\subsection{Erhalt des Vertrauens in den Straßenverkehr} \label{ErhaltDesVertrauensInDenStrassenverkehr}

\begin{quote}
\glqq
Automatisiertes Fahren ist nur in dem Maße vertretbar, in dem denkbare Angriffe, insbesondere Manipulationen des 
IT-Systems oder auch immanente Systemschwächen nicht
zu solchen Schäden führen, die das Vertrauen in den Straßenverkehr nachhaltig erschüttern.\grqq\mbox{~\cite[S. 12]{ek}}
\end{quote}

Die Frage ist hier, welche Komponenten des Gesamtsystems anfällig gegenüber böswilligen IT-Angriffen von außen sind. Dies sind m.\,E. weniger 
die öffentlichen Dienste oder Leitsysteme, da hier zwar Fehlfunktionen oder Ausfälle möglicherweise zu Verkehrschaos, nicht aber unmittelbar zu 
Gefahrensituationen führen. Anders verhält es sich
mit den Kommunikationssystemen der Fahrzeuge selbst. Sämtliche Car-to-Car- und Device-to-Car-Kommunikation\footnote{Mit \textit{Device-to-Car-Kommunikation} 
ist hier die Kommunikation zwischen dem Fahrzeug und einem Handheld-Device im Besitz des Fahrers, z.\,B. einem Mobiltelefon, beispielsweise zu dessen
Identifikation, gemeint. Diese Art der Kommunikation wird derzeit unter dem Industriestandard \textit{Bluetooth}~\cite{bt} ausgeführt.} dürfte daher ausschließlich verschlüsselt 
und sicher authentifiziert stattfinden, damit eine unerwünschte Einflußnahme Dritter auf ein bestimmtes Fahrzeug ausgeschlossen ist.

\subsection{Datenhoheit der Verkehrsteilnehmer} \label{DatenhoheitDerVerkehrsteilnehmer}

\begin{quote}
\glqq
Erlaubte Geschäftsmodelle, die sich die durch automatisiertes und vernetztes Fahren entstehenden, für die Fahrzeugsteuerung 
erheblichen oder unerheblichen Daten zunutze
machen, finden ihre Grenze in der Autonomie und Datenhoheit der Verkehrsteilnehmer.
Fahrzeughalter oder Fahrzeugnutzer entscheiden grundsätzlich über Weitergabe und
Verwendung ihrer anfallenden Fahrzeugdaten. Die Freiwilligkeit solcher Datenpreisgabe
setzt das Bestehen ernsthafter Alternativen und Praktikabilität voraus. Einer normativen
Kraft des Faktischen, wie sie etwa beim Datenzugriff durch die Betreiber von Suchmaschinen oder sozialen Netzwerken vorherrscht, 
sollte frühzeitig entgegengewirkt werden.\grqq\mbox{~\cite[S. 12]{ek}}
\end{quote}

\subsection{Internationale Standardisierung der Protokoll- und Dokumentationspflichten} \label{InternationaleStandardisierungDerProtokollUndDokumentationspflichten}

\begin{quote}
\glqq
Es muss klar unterscheidbar sein, ob ein fahrerloses System genutzt wird oder ein Fahrer
mit der Möglichkeit des \glqq Overrulings\grqq\ Verantwortung behält. Bei nicht fahrerlosen Systemen muss die Mensch/Maschine-Schnittstelle 
so ausgelegt werden, dass zu jedem Zeitpunkt klar geregelt und erkennbar ist, welche Zuständigkeiten auf welcher Seite liegen,
insbesondere auf welcher Seite die Kontrolle liegt. Die Verteilung der Zuständigkeiten
(und damit der Verantwortung) zum Beispiel im Hinblick auf Zeitpunkt und Zugriffsregelungen sollte dokumentiert und gespeichert werden. 
Das gilt vor allem für Übergabevorgänge zwischen Mensch und Technik. Eine internationale Standardisierung der Übergabevorgänge 
und der Dokumentation (Protokollierung) ist anzustreben, um angesichts der
grenzüberschreitenden Verbreitung automobiler und digitaler Technologien die Kompatibilität der Protokoll- oder 
Dokumentationspflichten zu gewährleisten.\grqq\mbox{~\cite[S. 13]{ek}}
\end{quote}

Dieses Postulat, nämlich die Notwendigkeit zum Führen eines standardisierten \glqq Logbuchs\grqq\ im Fahrzeug, knüpft unmittelbar an 
\ref{Produkthaftung}, also das Thema \textit{Produkthaftung}, an. Auf internationaler Ebene ist bei der Etablierung entsprechender Standards 
und Zertifizierungsstellen mit den üblichen Friktionen und Verzögerungen zu rechnen, da nicht nur Hersteller- sondern auch Länder- oder 
sonstige regionale Interessen berücksichtigt un in Einklang gebracht werden müssen.

\subsection{Vermeidung abrupter Übergabevorgänge} \label{VermeidungAbrupterUebergabevorgaenge}

\begin{quote}
\glqq
Software und Technik hochautomatisierter Fahrzeuge müssen so ausgelegt werden, dass
die Notwendigkeit einer abrupten Übergabe der Kontrolle an den Fahrer (\glqq Notstand\grqq)
praktisch ausgeschlossen ist. Um eine effiziente, zuverlässige und sichere Kommunikation zwischen Mensch und Maschine zu 
ermöglichen und Überforderung zu vermeiden,
müssen sich die Systeme stärker dem Kommunikationsverhalten des Menschen anpassen
und nicht umgekehrt erhöhte Anpassungsleistungen dem Menschen abverlangt werden.\grqq\mbox{~\cite[S. 13]{ek}}
\end{quote}

Dieser Punkt spricht unmittelbat das Thema \textit{Benutzerinterface} an.\footnote{Wer gelegentlich einen Mietwagen benutzt weiß, wie 
umständlich es sein kann, herauszufinden, wie man in einem nicht vertrauten Fahrzeugmodell auch nur das Licht einschaltet. Auf die
Benutzerinterface-Designer kommen im Kontext des \glqq hochautomatisierten Fahrens\grqq\ in den kommenden Jahren noch ungleich größere
Herausforderungen zu.}

\subsection{Aufbau eines Szenarienkatalogs} \label{AufbauEinesSzenarienkatalogs}

\begin{quote}
\glqq
Lernende und im Fahrzeugbetrieb selbstlernende Systeme sowie ihre Verbindung zu zentralen Szenarien-Datenbanken 
können ethisch erlaubt sein, wenn und soweit sie Sicherheitsgewinne erzielen. Selbstlernende Systeme dürfen nur dann eingesetzt werden, wenn
sie die Sicherheitsanforderungen an fahrzeugsteuerungsrelevante Funktionen erfüllen
und die hier aufgestellten Regeln nicht aushebeln. Es erscheint sinnvoll, relevante Szenarien an einen zentralen 
Szenarien-Katalog einer neutralen Instanz zu übergeben, um
entsprechende allgemeingültige Vorgaben, einschließlich etwaiger Abnahmetests zu erstellen.\grqq\mbox{~\cite[S. 13]{ek}}
\end{quote}



\subsection{Autonomes Überführen in sicheren Grundzustand in Notsituationen} \label{AutonomesUeberfuehrenInSicherenGrundzustandInNotsituationen}

\begin{quote}
\glqq
In Notsituationen muss das Fahrzeug autonom, d.h. ohne menschliche Unterstützung, in
einen \glqq sicheren Zustand\grqq\ gelangen. Eine Vereinheitlichung insbesondere der Definition
des sicheren Zustandes oder auch der Übergaberoutinen ist wünschenswert.\grqq\mbox{~\cite[S. 13]{ek}}
\end{quote}

Definition Notsituation. Sicherster Zustand ist Stand außerhalb des Verkehrs.

\subsection{Fahrausbildung} \label{Fahrausbildung}

\begin{quote}
\glqq
Die sachgerechte Nutzung automatisierter Systeme sollte bereits Teil der allgemeinen digitalen Bildung sein. Der sachgerechte 
Umgang mit automatisierten Fahrsystemen sollte
bei der Fahrausbildung in geeigneter Weise vermittelt und geprüft werden.\grqq\mbox{~\cite[S. 13]{ek}}
\end{quote}

Was passiert, wenn Skills, die zum manuellen Fahren benötigt werden, nicht mehr gelehrt werden?

\newpage

\cleardoublepage
\section{Schlussbemerkung}

\newpage

\cleardoublepage
\begin{thebibliography}{99}
\addcontentsline{toc}{section}{Literatur}

\bibitem{bt} Bluetooth Special Interest Group (SIG), Inc., https://www.bluetooth.com/, Stand Januar 2019

\bibitem{ek} Ethik-Kommission, Bundesministerium für Verkehr und digitale Infrastruktur, Automatisiertes und
Vernetztes Fahren, 2017

\bibitem{site1} PRESSEMITTEILUNG - 157/2016 Auftaktsitzung der Ethik-Kommission zum automatisierten Fahren, Stand 30.12.2018, \url{https://www.bmvi.de/SharedDocs/DE/Pressemitteilungen/2016/157-dobrindt-ethikkommission.html}


\end{thebibliography}

\newpage


\end{document}